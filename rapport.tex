\documentclass[a4paper, 12pt]{article}


\usepackage[utf8]{inputenc}
\usepackage[french]{babel}

\begin{document}
\tableofcontents

\title{INFO-F-201 – Systèmes d’exploitations\\Rapport du projet de programmation système en C\\TinyDB}
\author{Adegnon Kokou\\501910}
\maketitle{}
\section{Description}

Le projet TinyDB a pour but de simuler 4 commandes primaire exécuter sur une base de donnée, "select", "insert","update","delete".
le programme est structuré en plusieurs structures, une structure pour stocker les informations sur les étudiants, une autre pour représenter la base de donnée.
Ce dernier structure est sous la forme d'une liste dynamique avec une taille logique qui permet d’accéder aux éléments par indice et une taille réelle représente l'espace réel
que prend la liste dynamique. A cela s'ajoutent des fonctions pour agir sur la base de donnée et des fonctions pour exécuter les commandes.

Mon programme ne gère pas les erreurs typographique au sein des commandes, de plus il ne peut qu’exécuter une commande à la fois et pour finir les résultats d'une commande sont écrit 
dans le dossier logs qui doit être créé au préalable.

\section{Difficultés et solutions}

	\subsection{Ajout de Thread}
L'utilisation de thread dans ce projet était ,selon moi, la partie la plus difficile
du travail. En effet, avant d’implémenter les threads, il fallait penser 
l'architecture du projet. De prime abord, je me dirigeais vers une liste dynamique trié qui m'aurait permis d'avoir un thread qui
ne s'occupe que du tri après un "insert", un "delete" ou un "upgrade" mais la difficulté se trouvais dans le choix de l'option à trier.
J'ai opté pour une simple liste dynamique sans aucun trie.

Une autre fonctionnalité que les threads auraient permis, est le parallélisme des commandes lorsqu'ils sont exécutés avec une redirection de fichier.
Cette approche permet d’exécuter une énorme quantité de commande "en même temps". Toutefois le but d’exécuter des commandes à partir d'un fichier est de pouvoir
automatiser certaines actions séquentiellement.

La dernière utilisation des threads et mon implémentation réelle a été d'avoir des threads qui exécutent un "select",en parcourant linéairement la liste dynamique et sur différent morceau de la base de donnée pour ajouter le résultat a une autre base de donnée temporaire. Pour ce faire il me fallait délimiter les zones qu'allait traiter 
chaque threads et ajouter l’étudiante s'il respectait les critères de sélection, donc l’écriture dans la base de donnée temporaire doit être entourer
de mutex pour empêcher l’écriture en concurrence sur un même indice ou d'autre problème auquel je n'aurais 
pas penser suite à cet accès en concurrence à l’écriture de chaque thread.

Les threads ne sont lancés que pour les grandes bases de donnée ayant un nombre d’étudiant strictement supérieur à 999 998. Étant donné que les threads prennent un certain temps à ce lancer, j'ai jugé inutile de faire une recherche linéaire avec plusieurs threads si la base de donnée n’était pas assez grande pour rentabiliser ce temps. Une des difficultés qui s'est ajoutées après cela était le choix
du nombre de threads en sachant qu'il fallait un nombre qui soit assez grand pour être efficaces et surtout un nombre qui soit le plus grand commun diviseur avec le nombre 
d’étudiants présent dans la base de donnée  pour permettre une répartition équitable entre chaque thread. 

J'ai opté pour une solution très simpliste, qui a été de choisir un nombre arbitraire de thread et de couper ma liste de manière équitable pour aux finale rajouter dans le dernier thread les éléments qui ne sont rentrés dans aucun des threads. En plus à cella, j'ai rajouté des mutex pour un meilleur contrôle des sections critiques.


	\subsection{Suppression dans la base de donnée}

Exécuter la commande "delete" dans la base de donnée est assez compliquer sur une liste dynamique. L’implémentation classique pour une liste chaîné dynamique se compose d'un pointeur vers le
suivant et de l'élément à stocké donc pour faire une suppression à un endroit quelconque il suffit d’accéder à l’élément précédent et de faire pointer le pointeur vers le suivant vers le suivant de l’élément à supprimer. Cependant,dans mon cas, il n'y a pas ce pointeur donc la suppression d'un élément risque de laisser des trous dans la liste.

Mon implémentation du "delete" est la suivante, parcourir la liste à l'envers, exécuter un swap entre le dernier élément de la liste et l’élément à supprimer et pour finir incrémenter un compteur qui m'indique le dernier éléments puisque le dernier élément ne peut plus être swapper et dois surtout être supprimé.
Après avoir parcouru toute la liste, il suffit de soustraire, de la taille de la liste, le compteur qui indique le dernier élément.





\end{document}
